\documentclass[11pt, twoside, reqno]{book}
\usepackage{amssymb, amsthm, amsmath, amsfonts}
\usepackage[htt]{hyphenat}
\usepackage{graphicx}
\usepackage{color}
\usepackage{hyperref}
\usepackage{verbatim}
\usepackage{ wasysym }
\usepackage[toc,page]{appendix}
\appendixpageoff
\usepackage[leftmargin = 1in, rightmargin = 0in, vskip = 0in]{quoting}
\usepackage{microtype}
\usepackage{bardtex}
\biboption{amsrefs}
\styleoption{seniorproject}
\usepackage{listings}
\definecolor{codegreen}{rgb}{0,0.6,0}
\definecolor{codegray}{rgb}{0.5,0.5,0.5}
\definecolor{codepurple}{rgb}{0.58,0,0.82}
\definecolor{backcolour}{rgb}{0.95,0.95,0.98}
\lstdefinestyle{mystyle}{
    commentstyle=\color{codegreen},
    keywordstyle=\color{magenta},
    stringstyle=\color{codepurple},
    basicstyle=\ttfamily\footnotesize,
    backgroundcolor=\color{backcolour},
    frame=lrbt,
    breakatwhitespace=false,      
    framexleftmargin=8pt, 
    framexrightmargin=8pt,   
    framextopmargin=6pt,
    framexbottommargin=6pt,
    breaklines=true,                 
    keepspaces=false,                 
    numbersep=0pt,                  
    showspaces=false,                
    showstringspaces=false,
    showtabs=false,                  
    tabsize=1
}
\lstset{style=mystyle}
\newcommand{\blockquotespacing}{\blockspaced}
\newcommand{\prose}{.25in}
\newcommand{\poetry}{0in}
\newcommand{\singlespaced}{\setstretch{1}\vspace{\baselineskip}}
\newcommand{\blockspaced}{\setstretch{1.3}\vspace{\baselineskip}}
\newcommand{\doublespaced}{}
\newenvironment{blockquote}[1][\prose]{\setlength{\parindent}{#1}\begin{quoting}\blockquotespacing}{\end{quoting}}
\begin{document}
\startmain
\chapter{Experiment}

\section{Experimental Design}

\textit{Note: All experimentation was conducted with IRB approval.}

We conducted a survey to determine a baseline of human scores for passages by affect. There were two versions of the survey; both had the same first twelve passages, and a different set of the latter twelve passages. The first twelve passages were gathered by a sentence generator and served as unaltered text. The second set of twelve passages was gathered by feeding the first twelve through one of our replacement models; one version of the survey had all passages go through the arbitrary replacement model, and the other had all passages go through the targeted replacement model. For each passage, we asked the participants to give a score from one to ten in each of the following categories: how natural the passage sounded, how angry the passage sounded, how sad the passage sounded, how joyful the passage sounded, and how fearful the passage sounded.

This was conducted over the internet via Google Forms. As Forms does not support A/B testing, we embedded the surveys in a webpage using a \href{https://github.com/colehollant/quarantine-js}{microframework I built} to programmatically provide one of the two versions with equal probability. We received 30 responses across the two surveys, with an even split of 15 responses for each version.

\section{Results}

We conducted an independent samples t test on the control results for each category (scores for natural, anger, sadness, joy, fear) to gain insight on the difference between the two groups. Each category has a null hypothesis of the two groups having equal means; each category has the same 12 prompts, and we run the t test on each prompt. The ``natural'' category had one out of twelve p values below our $\alpha = 0.05$, thus in all other cases we fail to reject the null hypothesis—we are unable to conclude that the two groups are significantly different. Similarly, the ``anger'' category had one out of twelve, ``sadness'' had two out of twelve, ``joy'' had one out of twelve, and ``fear'' had two out of twelve p values less than $\alpha = 0.05$.

\subsection{Comparison of groups}


\subsubsection{\textbf{Neutral Questions}}

\begin{tabular}{|l|}
\hline
    question \\ \hline
    \texttt{01.}\hspace{8pt} When motorists sped in and out of traffic,\\\hspace{24pt} all she could think of was those in need of a transplant. \\ \hline
    \texttt{02.}\hspace{8pt} He drank life before spitting it out. \\ \hline
    \texttt{03.}\hspace{8pt} The toy brought back fond memories of being lost in the rain forest. \\ \hline
    \texttt{04.}\hspace{8pt} Italy is my favorite country; in fact, I plan to spend two weeks there next year. \\ \hline
    \texttt{05.}\hspace{8pt} The blinking lights of the antenna tower came into focus just as I heard a loud snap. \\ \hline
    \texttt{06.}\hspace{8pt} I love bacon, beer, birds, and baboons. \\ \hline
    \texttt{07.}\hspace{8pt} She saw the brake lights, but not in time. \\ \hline
    \texttt{08.}\hspace{8pt} They say that dogs are man's best friend, \\\hspace{24pt} but this cat was setting out to sabotage that theory. \\ \hline
    \texttt{09.}\hspace{8pt} The tart lemonade quenched her thirst, but not her longing. \\ \hline
    \texttt{10.}\hspace{8pt} He was surprised that his immense laziness was inspirational to others. \\ \hline
    \texttt{11.}\hspace{8pt} They got there early, and they got really good seats. \\ \hline
    \texttt{12.}\hspace{8pt} You can't compare apples and oranges, but what about bananas and plantains? \\ \hline
\end{tabular}
\vspace{16pt}


\subsubsection{\textbf{Natural Scores}}

\begin{tabular}{|l|l|l|l|l|l|l|l|}
\hline
    question & p value & mean difference & upper CI & lower CI & df & Cohen's d & effect size \\ \hline
    1. & 0.48079 & 0.57143 & 1.28675 & -0.14389 & 26 & 0.27037 & small \\ \hline
    2. & 0.94782 & 0.07143 & 0.13751 & 0.00535 & 26 & 0.02498 & small \\ \hline
    3. & 0.34479 & 0.71429 & 1.67654 & -0.24796 & 26 & 0.3637 & medium  \\ \hline
    4. & 0.72381 & 0.28571 & 0.64294 & -0.07151 & 26 & 0.13502 & small \\ \hline
    5. & 0.49097 & -0.5 & -1.19866 & 0.19866 & 26 & -0.26407 & small  \\ \hline
    6. & 0.71533 & 0.28571 & 0.65443 & -0.083 & 26 & 0.13936 & small \\ \hline
    7. & 0.50865 & -0.42857 & -1.09875 & 0.2416 & 26 & -0.2533 & small \\ \hline
    8. & 0.37381 & -0.57143 & -1.47636 & 0.3335 & 26 & -0.34203 & small \\ \hline
    9. & 0.42139 & -0.78571 & -1.60264 & 0.03122 & 26 & -0.30877 & small \\ \hline
    10. & 0.70986 & 0.35714 & 0.73329 & -0.01901 & 26 & 0.14217 & small \\ \hline
    11. & \textbf{0.03027} & 1.57143 & 3.86317 & -0.72031 & 26 & 0.8662 & large \\ \hline
    12. & 0.30227 & 0.85714 & 1.90961 & -0.19533 & 26 & 0.3978 & medium \\ \hline
\end{tabular}
\vspace{16pt}

\subsubsection{\textbf{Anger Scores}}

\begin{tabular}{|l|l|l|l|l|l|l|l|}
\hline
    question & p value & mean difference & upper CI & lower CI & df & Cohen's d & effect size  \\ \hline
    1. & 0.14972 & 1.21429 & 2.69871 & -0.27014 & 26 & 0.56106 & medium \\ \hline
    2. & 0.07658 & -1.57143 & -3.41563 & 0.27277 & 26 & -0.69704 & small \\ \hline
    3. & 0.65964 & -0.14286 & -0.58836 & 0.30265 & 26 & -0.16838 & small \\ \hline
    4. & 0.45892 & -0.14286 & -0.89467 & 0.60895 & 26 & -0.28416 & small \\ \hline
    5. & 0.13268 & -0.64286 & -2.19515 & 0.90944 & 26 & -0.58671 & small \\ \hline
    6. & 0.06944 & -0.28571 & -2.17944 & 1.60801 & 26 & -0.71576 & small \\ \hline
    7. & 0.8948 & -0.07143 & -0.20497 & 0.06211 & 26 & -0.05047 & small \\ \hline
    8. & 0.45384 & -0.64286 & -1.40329 & 0.11757 & 26 & -0.28742 & small \\ \hline
    9. & 0.53799 & -0.28571 & -0.90982 & 0.3384 & 26 & -0.23589 & small \\ \hline
    10. & 0.18126 & -0.42857 & -1.80229 & 0.94514 & 26 & -0.51922 & small \\ \hline
    11. & 0.14574 & -0.71429 & -2.214 & 0.78543 & 26 & -0.56684 & small \\ \hline
    12. & \textbf{0.04879} & -0.85714 & -2.9245 & 1.21022 & 26 & -0.78139 & small \\ \hline
\end{tabular}
\vspace{16pt}

\subsubsection{\textbf{Sadness Scores}}

\begin{tabular}{|l|l|l|l|l|l|l|l|}
\hline
    question & p value & mean difference & upper CI & lower CI & df & Cohen's d & effect size  \\ \hline
    1. & 0.45453 & -0.5 & -1.25926 & 0.25926 & 26 & -0.28697 & small \\ \hline
    2. & 0.66341 & -0.35714 & -0.79737 & 0.08308 & 26 & -0.16639 & small  \\ \hline
    3. & 0.05728 & -1.78571 & -3.77509 & 0.20366 & 26 & -0.75191 & small  \\ \hline
    4. & 0.14936 & -0.21429 & -1.70007 & 1.2715 & 26 & -0.56157 & small \\ \hline
    5. & 0.68706 & 0.21429 & 0.62166 & -0.19309 & 26 & 0.15397 & small  \\ \hline
    6. & 0.7837 & -0.07143 & -0.34878 & 0.20592 & 26 & -0.10483 & small \\ \hline
    7. & 0.92345 & 0.07143 & 0.16845 & -0.02559 & 26 & 0.03667 & small \\ \hline
    8. & 0.06638 & -0.71429 & -2.63061 & 1.20204 & 26 & -0.7243 & small \\ \hline
    9. & 0.08258 & -1.28571 & -3.09128 & 0.51985 & 26 & -0.68244 & small \\ \hline
    10. & 1.0 & 0.0 & 0.0 & 0.0 & 26 & 0.0 & small  \\ \hline
    11. & \textbf{0.03373} & -0.78571 & -3.02749 & 1.45606 & 26 & -0.84731 & small \\ \hline
    12. & \textbf{0.03594} & -0.57143 & -2.78369 & 1.64084 & 26 & -0.83616 & small \\ \hline
\end{tabular}
\vspace{16pt}

\subsubsection{\textbf{Joy Scores}}

\begin{tabular}{|l|l|l|l|l|l|l|l|}
\hline
    question & p value & mean difference & upper CI & lower CI & df & Cohen's d & effect size  \\ \hline
    1. & 0.15908 & -0.64286 & -2.09261 & 0.8069 & 26 & -0.54796 & small \\ \hline
    2. & 0.91834 & 0.07143 & 0.17495 & -0.0321 & 26 & 0.03913 & small  \\ \hline
    3. & 0.71771 & -0.35714 & -0.72263 & 0.00834 & 26 & -0.13814 & small  \\ \hline
    4. & 0.73313 & -0.21429 & -0.55893 & 0.13036 & 26 & -0.13026 & small \\ \hline
    5. & 0.81512 & 0.07143 & 0.30764 & -0.16478 & 26 & 0.08928 & small  \\ \hline
    6. & 0.86097 & -0.14286 & -0.31974 & 0.03403 & 26 & -0.06686 & small \\ \hline
    7. & \textbf{0.03103} & -0.28571 & -2.56607 & 1.99464 & 26 & -0.86189 & small \\ \hline
    8. & 0.57205 & 0.5 & 1.07228 & -0.07228 & 26 & 0.2163 & small \\ \hline
    9. & 0.06454 & 1.07143 & 3.00174 & -0.85889 & 26 & 0.72959 & large \\ \hline
    10. & 0.42582 & -0.71429 & -1.52335 & 0.09478 & 26 & -0.3058 & small  \\ \hline
    11. & 0.93493 & -0.07143 & -0.15386 & 0.01101 & 26 & -0.03116 & small \\ \hline
    12. & 0.52145 & 0.64286 & 1.29277 & -0.00705 & 26 & 0.24564 & small \\ \hline
\end{tabular}
\vspace{16pt}

\subsubsection{\textbf{Fear Scores}}

\begin{tabular}{|l|l|l|l|l|l|l|l|}
\hline
    question & p value & mean difference & upper CI & lower CI & df & Cohen's d & effect size  \\ \hline
    1. & 0.64988 & 0.35714 & 0.81638 & -0.1021 & 26 & 0.17358 & small \\ \hline
    2. & 0.7244 & 0.21429 & 0.57071 & -0.14214 & 26 & 0.13471 & small  \\ \hline
    3. & 0.75466 & 0.21429 & 0.53011 & -0.10154 & 26 & 0.11937 & small  \\ \hline
    4. & \textbf{0.03103} & -0.28571 & -2.56607 & 1.99464 & 26 & -0.86189 & small \\ \hline
    5. & 0.325 & -0.85714 & -1.86036 & 0.14608 & 26 & -0.37918 & small  \\ \hline
    6. & 0.13199 & -0.28571 & -1.8409 & 1.26947 & 26 & -0.58781 & small \\ \hline
    7. & 0.10962 & -1.5 & -3.15662 & 0.15662 & 26 & -0.62614 & small \\ \hline
    8. & 0.86424 & -0.14286 & -0.31553 & 0.02982 & 26 & -0.06526 & small \\ \hline
    9. & 0.36909 & -0.35714 & -1.27119 & 0.5569 & 26 & -0.34548 & small \\ \hline
    10. & 0.61713 & -0.21429 & -0.72027 & 0.2917 & 26 & -0.19124 & small  \\ \hline
    11. & \textbf{0.02071} & -0.5 & -2.96306 & 1.96306 & 26 & -0.93095 & small \\ \hline
    12. & 0.21367 & -0.28571 & -1.56047 & 0.98904 & 26 & -0.48181 & small \\ \hline
\end{tabular}
\vspace{16pt}

\subsection{Error Analysis}

Then, we chose a subset of scoring metrics (raw score, and LDA net scores for input and topic) to compare to the real scores determined by our participants. We will use relative error as a means of analysis. We will look at relative error averaged by category, and then subdivided by target and group.

\subsubsection{\textbf{LDA Input Score Error}}

\begin{tabular}{|l|l|l|l|l|l|}
\hline
    anger error & sadness error & joy error & fear error & mean error & std dev \\ \hline
    -0.85802 & -0.76072 & -0.70641 & -0.79792 & -0.78077 & 0.055211 \\ \hline
\end{tabular}
\vspace{16pt}

\subsubsection{\textbf{LDA Topic Score Error}}

\begin{tabular}{|l|l|l|l|l|l|}
\hline
    anger error & sadness error & joy error & fear error & mean error & std dev \\ \hline
    -0.53525 & -0.46815 & -0.36493 & -0.47174 & -0.46002 & 0.061043 \\ \hline
\end{tabular}
\vspace{16pt}

\subsubsection{\textbf{Raw Score Error}}
\begin{tabular}{|l|l|l|l|l|l|}
\hline
    anger error & sadness error & joy error & fear error & mean error & std dev \\ \hline
    -0.85647 & -0.85213 & -0.85312 & -0.8497 & -0.85286 & 0.002430 \\ \hline
\end{tabular}
\vspace{16pt}

We can see that the LDA net scores for topic has the least average error across the board, followed by the LDA input scores, with the raw score trailing at the end. This suggests that the LDA topic score may be the most appropriate scoring model that we offer, although it has the highest variance among error scores. We can further look at our results by group and affect target.

\subsubsection{\textbf{LDA Input Score Error}}

\begin{tabular}{|l|l|l|l|l|l|l|l|}
\hline
    group & target & anger err & sadness err & joy err & fear err & mean $|\text{err}|$ & std dev \\ \hline
    neutral &  & -0.96974 & -0.68386 & -0.84701 & -0.77157 & 0.81804 & 0.12113 \\ \hline
    control &  & -0.95562 & -0.95015 & -0.97787 & -0.79765 & 0.92032 & 0.08266 \\ \hline
    experimental & anger & -0.43843 & -0.48738 & -0.20712 & -0.98126 & 0.52855 & 0.32562 \\ \hline
    experimental & sadness & -0.56263 & 0.03343 & 0.67848 & -0.69709 & 0.49291 & 0.31204 \\ \hline
    experimental & joy & -1.0 & -1.0 & -0.39131 & -0.47553 & 0.71671 & 0.32892 \\ \hline
    experimental & fear & -0.79571 & -0.94346 & -0.89708 & -0.9416 & 0.89446 & 0.06924 \\ \hline
\end{tabular}
\vspace{16pt}

Here, we see that there the LDA input scoring has rather high error with the unaltered (neutral) and random-replacement (control) scores ($>80\%$), and a mix of high and low error among the the targeted replacement scores. We see high average error within the sentences with a target of ``fear'' and ``joy'' ($89\% \text{ and } 72\%$ respectively), with lower mean error for sentences targeting ``anger'' and ``sadness'' ($53\% \text{ and } 49\%$ respectively).

\subsubsection{\textbf{LDA Topic Score Error}}

\begin{tabular}{|l|l|l|l|l|l|l|l|l|}
\hline
    group & target & anger err & sadness err & joy err & fear err & mean $|\text{err}|$ & std dev  \\ \hline
    neutral &  & -0.51351 & -0.65078 & -0.49961 & -0.39885 & 0.51569 & 0.10354  \\ \hline
    control &  & -0.67275 & -0.45844 & -0.48453 & -0.49215 & 0.52697 & 0.09826  \\ \hline
    experimental & anger & -0.0917 & 0.13595 & 0.41926 & -0.81972 & 0.36666 & 0.33509  \\ \hline
    experimental & sadness & -0.67467 & -0.35851 & 0.1114 & 0.45561 & 0.40005 & 0.23349  \\ \hline
    experimental & joy & -0.93078 & -0.91615 & -0.88804 & -0.6649 & 0.84997 & 0.12465  \\ \hline
    experimental & fear & -0.5083 & -0.62584 & -0.64535 & 0.08332 & 0.46571 & 0.26201  \\ \hline
\end{tabular}
\vspace{16pt}

We see that the LDA topic scores are more consistent, largely hovering close to $40-50\%$, with sentences that targeted ``joy'' having much higher mean error at $\approx85\%$.

\subsubsection{\textbf{Raw Score Error}}

\begin{tabular}{|l|l|l|l|l|l|l|l|l|}
\hline
    group & target & anger err & sadness err & joy err & fear err & mean $|\text{err}|$ & std dev  \\ \hline
    neutral &  & -0.9763 & -0.91245 & -0.96795 & -0.89303 & 0.93743 & 0.04098  \\ \hline
    control &  & -0.9187 & -0.96874 & -0.98527 & -0.83505 & 0.92694 & 0.06748  \\ \hline
    experimental & anger & -0.51785 & -0.48804 & -0.3838 & -0.91828 & 0.57699 & 0.23467  \\ \hline
    experimental & sadness & -0.80069 & -0.43176 & -0.35877 & -0.29939 & 0.47265 & 0.22529  \\ \hline
    experimental & joy & -1.0 & -1.0 & -1.0 & -0.72623 & 0.93156 & 0.13689  \\ \hline
    experimental & fear & -0.5926 & -0.8712 & -0.80598 & -0.93233 & 0.80053 & 0.14791  \\ \hline
\end{tabular}
\vspace{16pt}

The raw score model gives scores rather similar to that of the LDA input scores, having high error ($>80\%$) for the neutral and control groups as well as the sentences targeting ``joy'' and ``fear,'' with lower levels of error ($40-60\%$) in the sentences targeting ``anger'' and ``sadness.''

\chapter{Discussion}

As per the relative error for each scoring model, we conclude that the LDA topic-scoring model gives the most human-like affect scores. Across all models, we saw a higher error for sentences targeting ``joy'' and ``fear'' as opposed to sentences targeting ``anger'' and ``sadness.'' This may be the result of issues with attempts at changing an underlying tone through word-choice alone, suggesting that theme dominates word-choice in human conceptions; it may also arise from limitations of our lexicons. There may be improvements by constructing our LDA model differently; a comparison between our current model and one based off a large corpus of pre-scored sentences may give more insight as to the appropriateness of our model—this has not been done partially due to the cost of gathering large-scale human data.

\section{Future Work}

Given both the state of the world at the end of this academic year and the scope of this project, there is material left undone. This section serves to address that, both in the sense of acknowledgment and with ideas for implementation if applicable.

There ought to be some LDA replacement model, perhaps targeting a topic with a score closest to the desired output; this could be done by preferring synonyms with high impact on the target topic. Some LDA playground feature could be interesting, such that a user could pass a corpus and vocabulary to receive a model; this would entail a greater degree of automation as well as some database for models, perhaps with some authentication system for saving remotely. Perhaps our model API should maintain a database connection in order to make less HTTP requests. A fuller featured frontend with more information in regards to analysis is in order; a user should be able to see the effect of each word on the output, and to be able to edit input manually with dropdown menus for word suggestions (based on synonyms). Despite having made a REST API as our CRUD wrapper, I think something like GQL may be more fitting so that we can have finer grained control over the requests. The output from the replacement models should have an option to score the generated text directly as opposed to copy-pasting the output, and on a similar note, we should have an option to save results for reference (via \texttt{localStorage}). Then, there should be several other UI niceties: light/dark mode should extend to fully support things like the documentation and the info popovers, there should be more info popovers across the site, we should suggest words in the thesaurus page if no results exist (via edit distance), and—for development—there should be some UI playground for mocking designs (this can be done by registering components globally, providing a textarea for the template, and compiling/rendering the results).
\end{document}